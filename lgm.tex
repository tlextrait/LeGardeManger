\documentclass[10pt]{book}
\usepackage[paperwidth=6in,
	paperheight=9in,
	includefoot,
	margin=1.25in]{geometry}

\usepackage[T1]{fontenc}
\usepackage[light,math]{iwona}

%%%%%%%%%%%%%%%%%%%%%%%%%%%%%%%
% MACROS

\newcommand{\degree}{$^\circ$}
\newcommand{\eacute}{\'{e}}
\newcommand{\egrave}{\`{e}}
\newcommand{\ecirc}{\^{e}}
\newcommand{\ucirc}{\^{u}}
%%%%%%%%%%%%%%%%%%%%%%%%%%%%%%%

\begin{document}

\pagestyle{plain}
\pagenumbering{roman}
\tableofcontents
\newpage
\pagenumbering{arabic}

\small

%%%%%%%%%%%%%%%%%%%%%%%%%%%%%%%%%%%%%%%%%%%%%%%%%%%%%%%%%%%%%%%%%%%%%%%%%%%%%%%%%%%
%	BASES
%%%%%%%%%%%%%%%%%%%%%%%%%%%%%%%%%%%%%%%%%%%%%%%%%%%%%%%%%%%%%%%%%%%%%%%%%%%%%%%%%%%

\chapter*{Bases}
\addcontentsline{toc}{part}{Bases}

% CHANTILLY CREAM
\newpage
\section*{Chantilly Cream}
\addcontentsline{toc}{section}{Chantilly Cream}
\subsection*{Ingredients}
	\begin{itemize}
		\item Heavy cream
		\item Sugar
	\end{itemize}
\subsection*{Procedure}
	\begin{enumerate}
		\item Beat the cream (an electric beater is highly recommended).  It helps if the cream and the bowl have been chilled in the freezer beforehand.
		\item When the cream develops soft peaks, add the sugar. Sifted confectioner's sugar is preferred.
		\item Additional flavors can be added with cocoa powder or vanilla extract.
	\end{enumerate}
\subsection*{Notes}
	\begin{itemize}
		\item Never add hot items into the cream.
		\item Make sure there is no water in the bowl or on the whisk. Water would prevent the cream from hardening.
		\item Beaten cream can be kept in the fridge for 3 days.
	\end{itemize}
	
% CHOUX PASTRY
\newpage\section*{Choux Pastry}
\addcontentsline{toc}{section}{Choux Pastry}
\subsection*{Ingredients}
	\begin{itemize}
		\item 150g water
		\item 100g flour
		\item 60g butter
		\item 40g sugar
		\item 5g baking powder
		\item 3 eggs
		\item Pinch of salt
	\end{itemize}
\subsection*{Procedure}
	\begin{enumerate}
		\item Pre-heat the oven to 200{\degree}C (392{\degree}F).
		\item Heat the water, salt, sugar, and butter in a pot.
		\item Bring the mixture to a boil; while waiting, sift the flour once.
		\item When the mixture is boiling, throw all the flour in at once and mix it.  It is very important to beat the flour quickly to create the smoothest batter.
		\item Take the pot off the stove, and start adding the eggs one by one.  Be sure that the egg is completely incorporated before adding more.
		\item Add the baking powder.
		\item Pipe, and put into the oven for 25 to 40 minutes depending on size.
	\end{enumerate}
	
% PASTRY CREAM
\newpage\section*{Pastry Cream}
\addcontentsline{toc}{section}{Pastry Cream}
\subsection*{Ingredients}
	\begin{itemize}
		\item 1L milk
		\item 200g sugar
		\item 100g flour
		\item 40g cocoa powder or 1 vanilla bean
		\item 6 egg yolks
	\end{itemize}
\subsection*{Procedure}
	\begin{enumerate}
		\item Beat the egg yolks and sugar until the mix turns white.
		\item If using cocoa powder, replace 40g of the flour with cocoa powder.
		\item Add the flour and cocoa powder. Add half the milk gradually while beating.
		\item Warm the rest of the milk in a pot for 3 to 4 minutes. If using vanilla, put the cut and scraped bean in the milk and let infuse for a few minutes, then remove the bean.
		\item Add the former mix in the pot and beat until it boils.
		\item Pour the cream in a tray to cool down.
	\end{enumerate}

% SHORTCAKE
\newpage\section*{Shortcake}
\addcontentsline{toc}{section}{Shortcake}
\subsection*{Ingredients}
	\begin{itemize}
		\item 125g flour
		\item 125g sugar
		\item 4 eggs
	\end{itemize}
\subsection*{Procedure}
	\begin{enumerate}
		\item Preheat oven to 180{\degree}C (355{\degree}F).
		\item Butter and flour a cake mold or cover a tray with parchment paper.
		\item Beat the sugar and eggs in a bain-marie or in a pot at very low temperature.
		\item Remove pot from stove and keep beating until cold.
		\item Sift the flour and fold the mix into it.
		\item Pour the mix into the cake mold or tray and bake for 25 to 30 minutes at 180{\degree}C (355{\degree}F).
	\end{enumerate}
\newpage

% SHORTCAKE
\newpage\section*{Shortcake - Roll-cake}
\addcontentsline{toc}{section}{Shortcake - Roll-cake}
\subsection*{Ingredients}
	\begin{itemize}
		\item 100g confectioner's sugar
		\item 75g flour
		\item 50g oil
		\item 40g milk
		\item 40g sugar
		\item 10g lemon juice
		\item 3g baking powder
		\item 3g salt
		\item 5 eggs		
	\end{itemize}
\subsection*{Procedure}
	\begin{enumerate}
		\item Preheat oven to 180{\degree}C (355{\degree}F).
		\item Separate the egg yolks from the whites.
		\item Mix the flour, baking powder, oil, milk, salt, egg yolks and sugar.
		\item Add the lemon juice.
		\item Whip the egg whites until soft peaks appear and add the sifted confectioner's sugar.
		\item Fold the whites into the former mix.
		\item Pour the mix into a tray covered by parchment paper.
		\item Bake for 13 minutes at 180{\degree}C (355{\degree}F), then change the temperature to 90{\degree}C (200{\degree}F) for 7 minutes.
		\item Let the shortcake fully cool down before using or rolling.
	\end{enumerate}
\newpage

% SHORTCRUST
\newpage\section*{Shortcrust}
\addcontentsline{toc}{section}{Shortcrust}
\subsection*{Ingredients}
	\begin{itemize}
		\item 250g flour
		\item 125g butter
		\item 1 egg yolk
		\item Pinch of salt
		\item Water
	\end{itemize}
\subsection*{Procedure}
	\begin{enumerate}
		\item Soften the butter by either resting it at room temperature or by microwaving it lightly.
		\item On a work surface or large cutting board, form a volcano with the flour.
		\item Put the butter and salt into the volcano.
		\item Softly sand with the finger tips.
		\item Dig a new hole in the flour and put in the egg yolk.
		\item Work by hand into a ball and add water or flour as necessary.
		\item Rest in the fridge for at least 30 minutes before using.
	\end{enumerate}
\newpage

% SWEETCRUST
\newpage\section*{Sweetcrust}
\addcontentsline{toc}{section}{Sweetcrust}
\subsection*{Ingredients}
	\begin{itemize}
		\item 250g flour
		\item 125g sugar
		\item 125g butter
		\item 10g vanilla sugar
		\item 5g baking powder
		\item 1 egg yolk
		\item Pinch of salt
		\item Water
	\end{itemize}
\subsection*{Procedure}
	\begin{enumerate}
		\item Soften the butter by either resting it at room temperature or by microwaving it lightly.
		\item On a work surface or large cutting board, form a volcano with the flour.
		\item Put the butter into the volcano.
		\item Softly sand with the finger tips.
		\item Dig a new hole in the flour and put in the salt, sugar, vanilla sugar, egg yolk and baking powder.
		\item Work by hand into a ball and add water or flour as necessary.
		\item Rest in the fridge for a couple hours before using.
	\end{enumerate}
\newpage

%%%%%%%%%%%%%%%%%%%%%%%%%%%%%%%%%%%%%%%%%%%%%%%%%%%%%%%%%%%%%%%%%%%%%%%%%%%%%%%%%%%
%	SWEET
%%%%%%%%%%%%%%%%%%%%%%%%%%%%%%%%%%%%%%%%%%%%%%%%%%%%%%%%%%%%%%%%%%%%%%%%%%%%%%%%%%%

\newpage
\chapter*{Sweet}
\addcontentsline{toc}{part}{Sweet}

% CANELES BORDELAIS

\newpage
\section*{Canel{\eacute}s Bordelais}
\addcontentsline{toc}{section}{Canel{\eacute}s Bordelais}
\subsection*{Ingredients}
	\begin{itemize}
		\item 1L milk
		\item 475g sugar
		\item 280g flour
		\item 100g dark rum
		\item 50g butter
		\item 10g vanilla extract
		\item 24 aluminum canel{\eacute} molds
		\item 4 ego yolks
		\item 2 eggs
		\item 2 vanilla beans
		\item Food-grade beeswax
	\end{itemize}
\subsection*{Procedure}
	\begin{enumerate}
		\item Boil the milk, butter, rum, vanilla extract and sliced vanilla beans.
		\item Shut the stove and let the mix infuse for 15 minutes.
		\item Beat the eggs and yolks with the sugar until pale and foamy.
		\item Remove the beans from the milk and scrape the seeds using your thumb and put those into the egg mix. Place the beans back into the milk.
		\item Heat the milk again and when it simmers, shut the stove and let it cool down for 2 minutes.
		\item Pour a fifth of the hot milk on the eggs and stir.
		\item Add the flour et mix well, then add the remainder of the milk.
		\item Rest the batter for 24 hours at room temperature.
		\item Mix for 2 minutes, making sure bubbles disappear and butter melts into the batter.
		\item Pre-heat oven to 275{\degree}C (527{\degree}F).
		\item Melt beeswax in a small pot and coat the molds.
		\item Fill the molds with batter up to 1mm from the top.
		\item Place the molds in the oven for 15 minutes.
		\item Change the temperature to 200{\degree}C (392{\degree}F) and leave in the oven for another 35 minutes.
	\end{enumerate}
\newpage

% CHOCOLATE MOUSSE

\newpage
\section*{Chocolate Mousse}
\addcontentsline{toc}{section}{Chocolate Mousse}
\subsection*{Ingredients}
	\begin{itemize}
		\item 250g dark chocolate
		\item 10g heavy cream
		\item 4 eggs
		\item Pinch of salt
	\end{itemize}
\subsection*{Procedure}
	\begin{enumerate}
		\item Separate the egg whites from the yolks.
		\item Beat the egg whites with a pinch of salt until hard peaks appear.
		\item Break the chocolate into small pieces and melt it in a bain-marie.
		\item Add the cream into the chocolate and stir.
		\item Add the yolks and stir until the chocolate is smooth.
		\item Fold the chocolate into the egg whites.
		\item Pour into cups and rest in the fridge for 4 hours.
	\end{enumerate}
\newpage

% CHOCOLATE SOUFFLE

\newpage
\section*{Chocolate Souffl{\eacute}}
\addcontentsline{toc}{section}{Chocolate Souffl{\eacute}}
\subsection*{Ingredients}
	\begin{itemize}
		\item 200g heavy cream
		\item 170g dark chocolate
		\item 100g sugar
		\item 5g corn starch
		\item 5g cocoa powder
		\item 4 eggs
	\end{itemize}
\subsection*{Procedure}
	\begin{enumerate}
		\item Butter and sugar some cups.
		\item Break the chocolate into small pieces.
		\item Separate the egg whites from the yolks.
		\item Beat the egg whites while adding the sugar until soft peaks appear.
		\item Put the cream into a pot and sift in the cocoa and corn starch.
		\item Heat the cream until it simmers.
		\item Remove from the stove and add the chocolate while beating strongly.
		\item Add the egg yolks and beat until smooth.
		\item Fold the egg whites into the chocolate.
		\item Pour the batter into the cups and keep in the fridge until ready to bake (freezing is okay).
		\item Pre-heat oven to 190{\degree}C (374{\degree}F) and bake for 10 to 12 minutes.
	\end{enumerate}
\newpage

% CLAFOUTIS

\newpage
\section*{Clafoutis}
\addcontentsline{toc}{section}{Clafoutis}
\subsection*{Ingredients}
	\begin{itemize}
		\item 375g cherries
		\item 300g milk
		\item 160g flour
		\item 120g sugar
		\item 4 eggs
		\item Vanilla extract
		\item Pinch of salt
	\end{itemize}
\subsection*{Procedure}
	\begin{enumerate}
		\item Pre-heat the oven to 200{\degree}C (392{\degree}F).
		\item Remove the pits from the cherries.
		\item Mix all the ingredients together.
		\item Butter and flour a tart mold then add the batter, distributing the cherries in the batter uniformly.
		\item Bake in the oven until no longer runny, color should be golden.
	\end{enumerate}
\newpage

% CREME BRULEE

\newpage
\section*{Cr{\ecirc}me Br{\ucirc}l{\eacute}e}
\addcontentsline{toc}{section}{Cr{\egrave}me Br{\ucirc}l{\eacute}e}
\subsection*{Ingredients}
	\begin{itemize}
		\item 500g heavy cream
		\item 80g sugar
		\item 5 egg yolks
		\item 1 $\frac{1}{2}$ gelatin sheet
		\item 1 vanilla bean
	\end{itemize}
\subsection*{Procedure}
	\begin{enumerate}
		\item Preheat oven to 100{\degree}C (212{\degree}F).
		\item Leave the gelatin to soak in cold water.
		\item Combine the yolks and half the sugar without mixing but stirring.
		\item Heat the cream, the other half of the sugar and the grated and sliced vanilla bean.
		\item As soon as the cream simmers, pour it onto the yolks.
		\item Put everything back into the pot on low heat and remove the bean.
		\item Stir the batter and remove from the stove as soon as a thin layer sticks to the spoon.
		\item Add the gelatin to the mix and stir.
		\item Pour into cups and bake for 50 minutes.
		\item Rest in the fridge for 5 hours.
		\item Caramelize sugar on top using a blowtorch.
	\end{enumerate}
\newpage

% CREME AUX OEUFS

\newpage
\section*{Cr{\egrave}me aux Oeufs}
\addcontentsline{toc}{section}{Cr{\egrave}me aux Oeufs}
\subsection*{Ingredients}
	\begin{itemize}
		\item 750g milk
		\item 250g cream
		\item 125g sugar
		\item 6 eggs
		\item Vanilla extract
	\end{itemize}
\subsection*{Procedure}
	\begin{enumerate}
		\item Mix the sugar and eggs in a large bowl.
		\item Boil the milk, cream and vanilla extract.
		\item When the milk comes to a boil, remove from stove and let it cool just until it is below 60{\degree}C (140{\degree}F).
		\item Pour the milk into the eggs and mix.
		\item Preheat the oven at 120{\degree}C (250{\degree}F).
		\item Bake for 45 minutes.
	\end{enumerate}
\newpage

% CREPES

\newpage
\section*{Cr{\ecirc}pes}
\addcontentsline{toc}{section}{Cr{\ecirc}pes}
\subsection*{Ingredients}
	\begin{itemize}
		\item 400g milk
		\item 280g flour
		\item 100g dark rum
		\item 80g sugar
		\item 3 eggs
		\item 1 table spoon vanilla extract
		\item Butter for the pan
	\end{itemize}
\subsection*{Procedure}
	\begin{enumerate}
		\item Mix the eggs, flour and sugar.
		\item Combine milk, rum and vanilla.
		\item Pour about 1/3 of the milk into the batter.
		\item Mix until flour clumps disappear, add more milk if too solid.
		\item Pour the rest of the milk slowly while mixing.
		\item Pre-heat the pan, butter it lightly and pour some batter to cover no more than 1/3 of the pan.
		\item Quickly tilt the pan to spread the batter evenly or use a scraper.
		\item Flip the cr{\ecirc}pe when ready and cook another few seconds.
		\item Butter the pan before each new cr{\ecirc}pe.
	\end{enumerate}
\subsection*{Notes}
	\begin{itemize}
		\item Use only whole milk.
		\item Cane sugar can improve flavor.
		\item Use a non-stick flat pan.
		\item A good cr{\ecirc}pe should be yellowish and have brown dots on one side.
		\item The second side cooks much faster than the first.
		\item At the right temperature, the first side should take about 20 seconds and the second side about 5 seconds.
		\item A trick to know that the first side is cooked is to wait until the very edge of the cr{\ecirc}pe starts becoming brown.
	\end{itemize}
\subsection*{Troubleshooting}
	\begin{itemize}
		\item If tiny holes appear in the cr{\ecirc}pe, it means the pan is too hot.
		\item If there are no bubbles, it means the pan is too cold.
		\item The first cr{\ecirc}pe is usually spoiled, undercooked or overcooked.
		\item The two sides of the cr{\ecirc}pe should look different.
	\end{itemize}
\newpage

% MADELEINES

\newpage
\section*{Madeleines}
\addcontentsline{toc}{section}{Madeleines}
\subsection*{Ingredients}
	\begin{itemize}
		\item 125g flour
		\item 100g sugar
		\item 90g butter
		\item 2 eggs
		\item 1 egg yolk
		\item 7g baking powder
		\item Vanilla extract or other aroma (optional)
	\end{itemize}
\subsection*{Procedure}
	\begin{enumerate}
		\item Beat the eggs and yolk with the sugar until foamy.
		\item Add vanilla extract or other aroma.
		\item Sift the flour and baking powder into the batter.
		\item Melt the butter and pour into the batter.
		\item Rest in the fridge for 2 hours.
		\item Butter and flour madeleine molds.
		\item Fill the molds to 3/4 with batter.
		\item Preheat oven to 200{\degree}C (382{\degree}F) if using paper or metal molds, or 190{\degree}C (374{\degree}F) if using silicone molds.
		\item Bake for 8 to 10 minutes if using paper or metal molds, or 12 to 15 minutes if using silicone molds.
	\end{enumerate}
\newpage

% MARBLE CAKE

\newpage
\section*{Marble Cake}
\addcontentsline{toc}{section}{Marble Cake}
\subsection*{Ingredients}
	\begin{itemize}
		\item 220g flour
		\item 220g confectionner's sugar
		\item 125g butter
		\item 30g whole milk
		\item 20g vanilla extract
		\item 10g cocoa powder
		\item 7g baking powder
		\item 3 eggs
	\end{itemize}
\subsection*{Procedure}
	\begin{enumerate}
		\item Pre-head oven to 160{\degree}C (320{\degree}F).
		\item Soften the butter in the micro-wave and mix it with the sugar and vanilla.
		\item Sift the flour and baking powder and fold in the eggs one by one.
		\item Take 1/3 of the flour mix and put it in a separate bowl.
		\item Add the cocoa and milk to this third of flour.
		\item Butter and flour a cake mold and pour in the chocolate batter and white batter alternatively.
		\item Bake for 40 minutes.
	\end{enumerate}
\newpage

% MILLE CREPE

\section*{Mille Cr{\ecirc}pe}
\addcontentsline{toc}{section}{Mille Cr{\ecirc}pe}
\subsection*{Ingredients}
	\begin{itemize}
		\item 10-15 cr{\ecirc}pes
		\item 500mL of pastry cream
		\item 500mL of heavy cream
	\end{itemize}
\subsection*{Procedure}
	\begin{enumerate}
		\item Make the cr{\ecirc}pes
		\item Cover the cr{\ecirc}pes with a cloth to prevent drying and cool down in the fridge a couple hours
		\item Make a pastry cream and let it cool down
		\item Whip the cream and fold into the pastry cream
		\item Save the best looking cr{\ecirc}pe and put it aside
		\item Stack the cr{\ecirc}pes up, spreading a thin layer of pastry cream on each cr{\ecirc}pe evenly covering the entire cr{\ecirc}pe
		\item Add the best looking cr{\ecirc}pe on top, sprinkle sugar on top and caramelize it gently with a torch without burning or drying the cr{\ecirc}pe
		\item Cover the cake with plastic wrap to ensure it remains moist
		\item Rest the wrapped cake in the fridge for 4 hours
	\end{enumerate}
\newpage

% PANNA COTTA

\newpage
\section*{Panna Cotta}
\addcontentsline{toc}{section}{Panna Cotta}
\subsection*{Ingredients}
	\begin{itemize}
		\item 500g heavy cream
		\item 50g sugar
		\item 3 gelatin sheets
		\item 1 vanilla bean
	\end{itemize}
\subsection*{Procedure}
	\begin{enumerate}
		\item Soak the gelatin sheets in cold water.
		\item Cut and scrape the vanilla bean.
		\item Heat the cream, sugar and vanilla bean and seeds.
		\item When simmering, remove from stove and add the gelatin.
		\item Mix well, pour into cups and rest in the fridge for 3 hours.
	\end{enumerate}
\newpage

% PANNA COTTA - RASPBERRY

\newpage
\section*{Panna Cotta - Raspberry}
\addcontentsline{toc}{section}{Panna Cotta - Raspberry}
\subsection*{Ingredients}
	\begin{itemize}
		\item 500g heavy cream
		\item 170g raspberries
		\item 50g sugar
		\item 20g raspberry sirup
		\item 10g raspberry liquor
		\item 3 gelatin sheets
		\item Mint leaves
	\end{itemize}
\subsection*{Procedure}
	\begin{enumerate}
		\item Soak the gelatin sheets in cold water.
		\item Cut the raspberries in halves.
		\item Heat the cream and sugar until hot.
		\item Add the raspberries, sirup and liquor then stir.
		\item When simmering, remove from stove and add the gelatin, stir gently but long enough to make sure the gelatin is properly mixed.
		\item Pour into cups and deposit mint leaves on the surface of each panna cotta.
		\item Rest in the fridge for 3 hours.
	\end{enumerate}
\newpage

% RICE PUDDING - FINNISH

\newpage
\section*{Rice Pudding - Finland}
\addcontentsline{toc}{section}{Rice Pudding - Finland}
\subsection*{Ingredients}
	\begin{itemize}
		\item 1L milk
		\item 250g white Japanese rice
		\item 250g water
		\item 250g heavy cream
		\item 40g sugar
		\item 20g butter
		\item 2 pinches salt
		\item Ground cinnamon
		\item Ground cardamom
		\item Sliced almonds
	\end{itemize}
\subsection*{Procedure}
	\begin{enumerate}
		\item Cook the rice and butter in water over medium-high heat, stirring frequently to prevent burning.
		\item Once all the water is absorbed, pour in half of the milk and reduce temperature to medium heat and keep stirring.
		\item Once all the milk is absorbed, add the remainder of the milk and keep on cooking and stirring.
		\item Once the rice has thickened, add the heavy cream and sugar.
		\item Add salt and remove from stove.
		\item Store in the fridge until cool, this also thickens the pudding.
		\item Serve pudding and top with ground cinnamon, cardamom and almonds.
	\end{enumerate}
\newpage

% SAVARIN AU RHUM

\newpage
\section*{Savarin au Rhum}
\addcontentsline{toc}{section}{Savarin au Rhum}
\subsection*{Ingredients - Savarin}
	\begin{itemize}
		\item 500g flour
		\item 250g eggs
		\item 10g salt
		\item 25g sugar
		\item 15g fresh yeast (5g dry yeast)
		\item 150g butter
	\end{itemize}
\subsection*{Ingredients - Syrup}
	\begin{itemize}
		\item 450g sugar
		\item 550g water
		\item 80g dark rum
	\end{itemize}
\subsection*{Procedure}
	\begin{enumerate}
		\item Combine flour, sugar and salt.
		\item Pour 3 table spoons of warm water on the fresh or dry yeast to activate it.
		\item Transfer the revived yeast into the flour mix and start mixing.
		\item Beat the eggs, then pour into the flour.
		\item Keep on mixing at high speed for another 5 to 10 minutes at least.
		\item Cover the batter with clear film while ensuring it touches the batter to prevent dehydration. Rest for one hour at room temperature. The batter should double in volume.
		\item Remove the film and scrape the batter off of it.
		\item Melt the butter, let it cool down a bit and pour into the batter while mixing.
		\item Lightly grease the savarin molds with butter.
		\item Fill the molds with batter as follows: up to half for large molds and up to one third for small molds.
		\item Preheat oven to 180{\degree}C (355{\degree}F) and let the batter rest in the molds for another half hour.
		\item Start preparing the syrup by heating the water and sugar.
		\item When the water comes to a boil, remove from stove and add the rum.
		\item Put the molds into the oven for 10-15 minutes for the small ones, 15-20 minutes for medium ones and 20 minutes for large ones.
		\item Remove from the molds and dip into the syrup.
		\item Serve the savarins, garnish with whipped cream.
	\end{enumerate}
\newpage

% TIRAMISU

\newpage
\section*{Tiramisu}
\addcontentsline{toc}{section}{Tiramisu}
\subsection*{Ingredients}
	\begin{itemize}
		\item 350g mascarpone
		\item 125g sugar
		\item 30-40 lady finger biscuits
		\item 10g vanilla sugar
		\item 4 eggs
		\item ~500g hot coffee
		\item Pinch of salt
		\item Cocoa powder
	\end{itemize}
\subsection*{Procedure}
	\begin{enumerate}
		\item Dip the lady fingers one by one into hot coffee (no more than 2 seconds) and cover a deep tray.
		\item Separate the egg whites and yolks.
		\item Beat the yolks, sugar, vanilla sugar until foamy.
		\item Fold the mascarpone into the batter.
		\item Beat the egg whites with a pinch of salt until hard peaks appear.
		\item Fold the whites into the batter.
		\item Pour the batter onto the lady fingers, filling the tray.
		\item Rest the tiramisu in the fridge for at least 4 hours.
	\end{enumerate}
\newpage

% WAFFLES - LIEGES

\newpage
\section*{Waffles (Lieges)}
\addcontentsline{toc}{section}{Waffles (Lieges)}
\subsection*{Ingredients}
	\begin{itemize}
		\item 500g flour
		\item 250g milk
		\item 200g butter
		\item 200g pearl sugar
		\item 20g fresh yeast (or 10g dry yeast)
		\item 20g vanilla sugar
		\item 3 eggs
		\item Cinnamon powder, honey, maple syrup... (optional)
		\item Pinch of salt
	\end{itemize}
\subsection*{Procedure}
	\begin{enumerate}
		\item If using fresh yeast, dip it in warm milk for a few minutes.
		\item Soften the butter in the micro-wave oven.
		\item Mix everything into a dough, except for the pearl sugar.
		\item Rest for 20 minutes (optional).
		\item The dough should nearly double in size. Now gently add the pearl sugar.
		\item Dough is ready to be cooked.
	\end{enumerate}
\newpage

%%%%%%%%%%%%%%%%%%%%%%%%%%%%%%%%%%%%%%%%%%%%%%%%%%%%%%%%%%%%%%%%%%%%%%%%%%%%%%%%%%%
%	SAVORY
%%%%%%%%%%%%%%%%%%%%%%%%%%%%%%%%%%%%%%%%%%%%%%%%%%%%%%%%%%%%%%%%%%%%%%%%%%%%%%%%%%%

\newpage
\chapter*{Savory}
\addcontentsline{toc}{part}{Savory}

% BRANDADE

\newpage
\section*{Brandade}
\addcontentsline{toc}{section}{Brandade}
\subsection*{Ingredients}
	\begin{itemize}
		\item 850g salt cod
		\item 650g potatoes
		\item 4 garlic cloves
		\item 150g olive oil
		\item 150g cream
		\item Salt, pepper
		\item Thyme, laurel, fresh parsley
	\end{itemize}
\subsection*{Procedure}
	\begin{enumerate}
		\item Desalt the cod in cold water for at least 12 hours, changing the water every couple hours.
		\item Drain the cod and weigh it. If it is not exactly 850g then use the table in the notes to calculate the correct amount of potatoes.
		\item Cut the cod into small pieces and place into a pot.
		\item Add thyme and laurel.
		\item Cook at low heat until it boils, then remove from stove.
		\item Pour everything into a sift and throw out the herbs.
		\item Crush the fish with a fork to obtain very small pieces.
		\item Peel, cut and boil the potatoes until soft.
		\item Mash the potatoes alone.
		\item Preheat the oven to 210{\degree}C (410{\degree}F).
		\item Mince the parsley.
		\item Heat all the olive oil and crushed garlic in a large pot.
		\item Allow the oil to heat for 2 minutes and add the fish.
		\item Now add the mashed potatoes.
		\item Mix in the pot while on the stove using a whisk until fairly homogeneous.
		\item Add the cream, salt, pepper, parsley and keep mixing.
		\item Remove from stove and pour all contents into a baking tray.
		\item Score the surface using a fork.
		\item Put into the oven for 20 minutes.
	\end{enumerate}
\subsection*{Notes}
	\begin{itemize}
		\item The most important is the ratio of cod to potatoes. If the amount of cod is not 850g, then the amount of potatoes has to be adjusted proportionally (see table below).
	\end{itemize}
	
	\begin{center}
	\begin{tabular}{l | r | r | r | r | r | r | r | r}
		\textbf{Cod} 		& 1kg 	& 900g & 850g & 800g & 750g & 700g & 650g & 600g\\
		\hline\hline
		\textbf{Potatoes} 	& 765g 	& 690g & 650g & 610g & 575g & 535g & 500g & 460g\\
		\hline
		\textbf{Olive oil} 	& 177g 	& 160g & 150g & 140g & 132g & 125g & 115g & 105g\\
		\hline
		\textbf{Cream}	 	& 177g 	& 160g & 150g & 140g & 132g & 125g & 115g & 105g\\
	\end{tabular}
	\end{center}
	
\newpage

% QUICHE

\newpage
\section*{Quiche}
\addcontentsline{toc}{section}{Quiche}
\subsection*{Ingredients}
	\begin{itemize}
		\item 4 eggs
		\item 250g heavy cream
		\item Shortcrust
		\item Bacon
		\item Grated cheese
		\item Dijon mustard
		\item Salt
	\end{itemize}
\subsection*{Procedure}
	\begin{enumerate}
		\item Follow the shortcrust recipe in the Bases chapter.
		\item Poke holes at the bottom of the crust using a fork.
		\item Pre-bake the shortcrust for a few minutes at 180{\degree}C (356{\degree}F).
		\item Coat the shortcrust with mustard using a food brush.
		\item Cut the bacon into small pieces, fry on a pan and add in the crust.
		\item Add grated cheese, preferably Swiss cheese, Emmental or Gruyere for stronger flavor.
		\item Beat the eggs and cream lightly and pour into the crust.
		\item Bake at 180{\degree}C (356{\degree}F) until light brown.
	\end{enumerate}
\newpage

% LOBSTER - BOILED

\newpage
\section*{Lobster - Boiled}
\addcontentsline{toc}{section}{Lobster - Boiled}
\subsection*{Ingredients}
	\begin{itemize}
		\item Live lobster
	\end{itemize}
\subsection*{Procedure}
	\begin{enumerate}
		\item Boil water in a large pot
		\item Put live lobster into boiling water
	\end{enumerate}
	
	\vspace{1cm}
	
	\begin{center}
	\begin{tabular}{r | l}
		Lobster weight & Cooking time\\
		\hline
		1 pound &	8 minutes\\
		1 1/4 pounds &	9-10 minutes\\
		1 1/2 pounds &	11-12 minutes\\
		1 3/4 pounds &	12-13 minutes\\
		2 pounds &	15 minutes\\
		2 1/2 pounds &	20 minutes\\
		3 pounds &	25 minutes\\
		5 pounds &	35-40 minutes\\
	\end{tabular}
	\end{center}
\newpage

% NITAMAGO EGGS

\newpage
\section*{Nitamago Eggs}
\addcontentsline{toc}{section}{Nitamago Eggs}
\subsection*{Ingredients}
	\begin{itemize}
		\item 50mL tsuyu
		\item 50mL cooking sake
		\item 25mL soy sauce
		\item 25mL mirin
		\item 1 teaspoon of ginger paste
		\item Eggs
		\item Pepper
	\end{itemize}
\subsection*{Procedure}
	\begin{enumerate}
		\item Pour enough water into a pot so it would cover all the eggs and bring to a boil.
		\item Gently add the eggs into the pot and cook for 7 minutes on medium heat.\footnote{Medium heat ensures the eggs don't move to much in the pot and prevents them from breaking. If they break then water will leak into them.}
		\item In a separate pan, boil the tsuyu, cooking sake, soy sauce, mirin and ginger paster, then rest and let the marinade cool.
		\item When the eggs are ready, remove from the water and put them into iced water.\footnote{This stops the cooking process instantly.}
		\item Rest the eggs at least 5 minutes and peel them.
		\item In a sealable bag, add the eggs and the cold marinade. Rest in the fridge for 1 or 2 days before consuming.
	\end{enumerate}
\newpage

%%%%%%%%%%%%%%%%%%%%%%%%%%%%%%%%%%%%%%%%%%%%%%%%%%%%%%%%%%%%%%%%%%%%%%%%%%%%%%%%%%%
%	METHODS
%%%%%%%%%%%%%%%%%%%%%%%%%%%%%%%%%%%%%%%%%%%%%%%%%%%%%%%%%%%%%%%%%%%%%%%%%%%%%%%%%%%

\newpage
\chapter*{Methods}
\addcontentsline{toc}{part}{Methods}

\newpage
\section*{Bain-Marie}
\addcontentsline{toc}{section}{Bain-Marie}

\indent\indent Bain-Marie is a cooking method that ensures food is cooked at a constant and low temperature. It is achieved by filling a pot with water, bringing it to a boil and adding a smaller pot on top. The food is cooked in the smaller pot. Because water evaporates at 100{\degree}C, this process guarantees the food doesn't heat over 100{\degree}C. Bain-Marie is used for cooking delicate foods such as chocolate, which could otherwise burn easily.

\newpage
\section*{Beating cream}
\addcontentsline{toc}{section}{Beating cream}

\indent\indent Always use 'heavy cream', 'whipping cream' or 'whole cream'. Any reduced-fat cream will not beat. Make sure the mixing bowl is cold and completely dry. For best results, place the bowl and whisk in the fridge for a few minutes before beating. Then beat either by hand or using a mixer. 

Once the cream is hard enough that the whisk slices through it, you can optionally add flavorings in solid form such as sugar, cocoa powder, vanilla etc. Finer grain powders are better, such as confectioner's sugar. Sifting those can also help prevent the cream from collapsing.

Be careful, it you beat the cream too much it will transform into butter! Always keep the cream in the fridge until ready to be used.

\newpage
\section*{Beating egg whites}
\addcontentsline{toc}{section}{Beating egg whites}

\indent\indent Beating egg whites is similar to beating heavy cream. Make sure the mixing bowl is cold and completely dry. Put in the egg whites, a pinch of salt and start mixing.

Egg whites can be mixed to different levels of hardness, and each recipe will require a specific hardness. When a recipe specifies 'hard peaks', it means you need to mix the egg whites until the whisk can slice through it. At that point, if you turn over the mixing bowl, the egg whites should remain stuck to the bottom and not fall out.

If the recipe require 'soft peaks', then it means the egg whites should create long peaks when you remove the whisk.

Egg whites can collapse very easily so if the recipe needs sugar to be added into the bowl, make sure it is confectioner's sugar and that it is sifted. Add only a little bit at a time, mix some more and repeat.

%\newpage
%\section*{Beating yolks and sugar}
%\addcontentsline{toc}{section}{Beating yolks and sugar}

%\newpage
%\section*{Choosing rum}
%\addcontentsline{toc}{section}{Choosing rum}

%\newpage
%\section*{Choosing vanilla}
%\addcontentsline{toc}{section}{Choosing vanilla}

\newpage
\section*{Cooking eggs}
\addcontentsline{toc}{section}{Cooking eggs}

\subsection*{Cooking times}

\begin{center}
\begin{tabular}{l | c}
 & Time \\
\hline
Poached egg & 5" \\
Nitamago egg & 7" \\
Soft boiled egg & 6" to 8" \\
Hard boiled egg & 10" to 12" \\
\end{tabular}
\end{center}

\subsection*{Soft scrambled eggs}

Whip the eggs a little and mix in a little bit of heavy milk. Keep folding on the pan with a spatula until no longer runny.

\subsection*{Perfect scrambled eggs (Gordon Ramsay)}

Drops eggs on a medium-high heat pan, along with a scoop of butter for each egg. Stir with a spatular, break the yolks and let them mix with the butter and egg whites. Keep stirring for 4-5 minutes and if the pan gets too hot then lift it off the stove for a few seconds. Before taking the eggs off the heat, add a dash of milk, sour cream or heavy cream.

%\newpage
%\section*{Melting chocolate}
%\addcontentsline{toc}{section}{Melting chocolate}

%%%%%%%%%%%%%%%%%%%%%%%%%%%%%%%%%%%%%%%%%%%%%%%%%%%%%%%%%%%%%%%%%%%%%%%%%%%%%%%%%%%
%	APPENDIX
%%%%%%%%%%%%%%%%%%%%%%%%%%%%%%%%%%%%%%%%%%%%%%%%%%%%%%%%%%%%%%%%%%%%%%%%%%%%%%%%%%%

\newpage
\chapter*{Appendix}
\addcontentsline{toc}{part}{Appendix}

\newpage
\section*{Temperature conversion}
\addcontentsline{toc}{section}{Temperature conversion}

\begin{center}
\begin{tabular}{r c l}
Celsius & Fahrenheit & Thermostat \\
\hline
0C & 32F & - \\
10C & 50F & - \\
20C & 68F & - \\
30C & 86F & - \\
40C & 104F & 1 \\
50C & 122F & - \\
60C & 140F & - \\
70C & 158F & 2 \\
80C & 176F & - \\
90C & 194F & 3 \\
\hline
100C & 212F & - \\
110C & 230F & - \\
120C & 248F & 4 \\
130C & 266F & - \\
140C & 284F & - \\
150C & 302F & 5 \\
160C & 320F & - \\
170C & 338F & - \\
180C & 356F & 6 \\
190C & 374F & - \\
\hline
200C & 392F & 7 \\
210C & 410F & - \\
220C & 428F & - \\
230C & 446F & 8 \\
240C & 464F & - \\
250C & 482F & - \\
260C & 500F & 9 \\
270C & 518F & - \\
280C & 536F & - \\
290C & 554F & 10 \\
\end{tabular}
\end{center}

\newpage
\section*{Conversions}
\addcontentsline{toc}{section}{Other conversions}
\subsection*{Gelatin}
1 gelatin sheet = 1g of gelatin powder

\subsection*{Yeast}
1g fresh yeast = 1/4g dry yeast

\newpage
\section*{About}
\addcontentsline{toc}{section}{About}

Written by Thomas Lextrait, January 17th, 2012 - 2014.\\\\
Thanks to,

Nhi Vo, Alex Kuang, Emma Lextrait, Eija Lextrait, Vincent Lextrait, Bernard Laurance.

\newpage
\section*{Websites}
\addcontentsline{toc}{section}{Websites}

\begin{tabular}{ l r }
http://tlextrait.com & Thomas Lextrait \\
http://tlextrait.svbtle.com & Blog\\\\
http://www.lacuisinedebernard.com & Bernard Laurance \\
http://www.marmiton.org & Marmiton\\
http://www.meilleurduchef.com & Le Meilleur du Chef\\
\end{tabular}

\newpage
\end{document}